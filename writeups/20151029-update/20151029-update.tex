\documentclass[letterpaper,10pt]{article}
\usepackage{fancyhdr}
\usepackage{amsmath}
\usepackage{amssymb}
\usepackage{bm}
\usepackage{graphicx}
\usepackage{enumerate}
\usepackage{caption}
% \usepackage{natbib}
\usepackage{subcaption}
\usepackage{hyperref}
\usepackage{enumerate}
\usepackage[margin=1in]{geometry}
\pagestyle{fancy}
\lhead{Jesse Mu, Andrew Francl}
\rhead{CSCI3341 Update 1}

\begin{document}

\section*{Imitating famous artists with convolutional neural networks}

We propose to develop a machine learning system modeled after an algorithm in a
recent paper by Gatys et al \cite{gatys15}. The algorithm in the paper
describes a convolutional neural network algorithm that enables any given
photograph to be recreated in the style of famous paintings. Specifically, the
algorithm is able to take a painting and learn its stylistic composition rather
than its content, then apply this stylistic information to a new image to
create a new version of the image that stylistically matches the original
painting. We plan on extending the original algorithm in the paper by training
our system to recreate images not in the style of a single image, but in the
style of an artist based on his/her representative artwork.

\subsection*{Implementation}

We plan on creating a command-line tool in Python, interfacing with the Caffe
\cite{jia14} deep learning framework via pycaffe, to first recreate the
algorithm demonstrated in the paper and then enable support for artist
replication. Open source implementations of the algorithm using Lua and Torch
do exist\footnote{e.g.\
  \href{https://github.com/jcjohnson/neural-style}{https://github.com/jcjohnson/neural-style}}
  but we'll be building an original implementation in Python to gain a better
  understanding of how the system works.

  Once base functionality is
  established, we plan on expanding upon the original algorithm in the paper by
  attempting to recreate images not in the style of a single famous painting
  but by the style of an artist oeuvre. This will require either 1) learning a
  stylistic representation of an artist's lifetime work by training a network
  on not just one, but many paintings, or 2) stochastically chosing a painting
  or set of paintings on which to recreate a picture. Art for given artists
  will be obtained via automated scraping of
  WikiArt\footnote{\href{http://www.wikiart.org/}{http://www.wikiart.org/}}, which provides a large database of
  artwork categorized by artist.

\paragraph{Milestone 1 (10/29)} Establish GitHub repository with writeups, background
information, and utility scripts. Create a Linux server set
up on the Caffe and pycaffe framework, from which we will do more intensive
computation. Download the VGG-19 Caffe model used in \cite{gatys15}. Write
utility script to scrape artwork from WikiArt, and download art for one or two
preliminary artists. (Done)
\paragraph{Milestone 2 (11/12)} Develop command line tool implementing base
functionality as described in the paper: recreating an image based
on the style of a single image.
\paragraph{Milestone 3 (12/3)} Enhanced functionality: develop support for
recreating an image based on preconfigured artist style representations. For
now we'll pick a couple of well known artists such as Van Gogh and Picasso.
Polish and finalize CLI tool.

\subsection*{Progress}
As described in Milestone 1, we've already set up the infrastructure required
to begin developing the original algorithm described in the paper. We wrote a
Python script to scrape artwork from WikiArt, and downloaded the entire corpus
of Van Gogh's work, nearly 2000 images. The script is extensible and can
download artwork from any artist.

Because our
laptops aren't particularly powerful, We purchased a temporary Linux server with
some student credit via DigitalOcean\footnote{\href{https://www.digitalocean.com/}{https://www.digitalocean.com/}}, and set up Caffe. We've done preliminary
investigation on the downloaded VGG-19 network on an iPython notebook,
including initializing the VGG network, loading images, creating a
(rudimentary) preprocessor, and calling one iteration of forward propagation
with an example image.

\subsection*{Deliverables}
Deliverables include a GitHub repository of all of the work we'll complete this
semester, including a pdf writeup of the steps taken in implementing our
system, the Python and Caffe code, and instructions and examples.

\subsection*{Helpful topics from the AI course}

\begin{itemize}
    \item Machine learning, and specifically artificial neural networks. We will be
    building particularly advanced models of ANNs to handle image processing
    tasks. In learning more about deep learning, a survey by LeCun, Bengio, and
    Hinton \cite{lecun2015}, and a review of representation learning by Bengio
    et al. \cite{bengio2012} will be helpful.
    \item Probability can be used to incorporate random elements into the
      processing of the images (e.g.\ controlling style versus content
      tradeoffs, perhaps randomly selecting paintings)
\end{itemize}

\subsection*{Challenges}

\begin{itemize}
    \item The principal challenge is not only understanding the paper's
      algorithm, but extending it to incorporate the stylistic content of a
      complete body of images. From a technical perspective, this involves
      training the CNN in charge of handling style with many images of a single
      artist, which will also require a fundamental understanding of how the
      network works.
    \item Conceptually, it's unclear whether the stylistic information derived
      from training the network on several images will result in a general
      sense of an artist's style, or a confusing and meaningless average of an
      artist's work. For example, some artists vary in style considerably over
      their careers. If the latter, we need to make decisions about
      how to obtain a sensible stylistic representation. It may be useful to
      limit the training images to a specific subset of similar paintings.
    \item One challenge will be optimizing computation time. CNNs are
      computationally intensive and can take significant time to train.
      Optimizations on both the software level, such as saving model
      parameters, and the hardware level, such as enabling CUDA GPU
      computation, must be taken into account when designing our system.
\textbf{Update} we've partially circumvented the hardware problem by renting a
moderately poweful Linux server.
\end{itemize}


\nocite{*} % Include all references in bib
\bibliographystyle{annotate}
\bibliography{20151029-update}{}

\end{document}
